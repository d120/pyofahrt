\documentclass[a4paper]{tudexercise}

\usepackage[ngerman]{babel}
\usepackage[utf8]{inputenc}

\usepackage{ltablex}

\title{Zimmerverteilung}
\subtitle{Diese Liste zeigt die Verteilung der Teilnehmer*innen auf die Zimmer an.}


\newcommand{\tabellenzeile}[3]{
#1 & #2 & #3 \\
}


\begin{document}

\maketitle
\vspace{20px}
\begin{tabularx}{\textwidth}{X|X|X}
Vorname & Nachname & Zimmer\\
\hline

\tabellenzeile{ {{ member.first_name }} }{ {{ member.last_name }} }{  \textcolor{red}{\textit{kein Raum zugeteilt}}  {{ member.room }} }


\tabellenzeile{ {{ member.first_name }} }{ {{ member.last_name }} }{  \textcolor{red}{\textit{kein Raum zugeteilt}}  {{ member.get_room }} }

\end{tabularx}

\end{document}
